\section{Viðauki}
Hér skal vera dagbók frá öllum í verkefninu .
Við ákváðum strax að gera hendi og vorum búnir að gera skýslu langt fram í tímann en vörurnar frá Danmörku virkuðu ekki eins og við héldum og kom mjög seint og seinkun þessi varð til vandræða.
Við ákváðum að gera hendi sem væri stillt með loftþrýstingi en vegna seinkunar á sendingunni frá Danmörku þá skiftum við um verkefni.
Nýja verkefnið er flokkari sem flokkar með lit. Við notum m&m til að láta hna flokka.
Sindri er að sjá um skýrluna og að finna nauðsynlegar upplýsingar á netinu, ég er að forrita og hanna tækið.
Við höldum áfram með okkar verk og nýtum okkur frelsið sem kennarinn gaf okkur og gerum sem mest heima.

\begingroup
\obeylines
\input{dagbok.txt}
\endgroup
\subsection{Kóði Arduino}
Hér hef ég includað kóðan frá arduino sem er forritunarmálið C. Þetta getið þið endurtekið fyrir php kóða sem þið vistið í möppuni php eða python í möppunni python
\subsection{Januar}
I Januar hofum vid verid ad vinna i skirslu gerdum auk minni modelum af hondinni. Vid vonumst til ad fa partana fra vex i byrjun februar svo vid getum byrjad alvoru honnun og samsetningu. Einnig hofum vid tekid ta akvordunn ad vid munum skrifa allann okkar koda fyrir tetta verkefni i C# tar sem tad er tungumalid sem okkur lidum badum best med. Aftura moti verdur fyrsta utgafa velmennising stjornad af venjulega VEX heilanum tannig ad vid munum turfa ad skrifa tad i robotC
\begingroup
\lstinputlisting[language=Octave]{../DHTtester_get_php_sql/DHTtester_get_php_sql.ino}
\endgroup

\section{Inngangur}
Hér skal gera lýsingu á verkefninu þ.e hvað,  hvernig og  hvaða forritunarmál, fyrir hverja og hvaða notagildi verkefnið hefur. Minnst 300 orð. Notagildi skiptir miklumáli, reynið að sjá fyrir ykkur hverjir geti notað vélmennið ykkar og í hvaða tilgangi.  Þá kemur í ljós að 500 orð er frekar lítið :-) Hér er gott að byrja á því að lesa til um Arduino en allt hjá þeim er open-sourse og svo er hægt að lesa sér til um efnið í útgefnum bókum sem "programming Arduino \cite{monk} Skoðið vel heimildaskrá og skránna mybib.bib. Hér er gott að lýsa högun kerfisins með orðum og mynd sem þið getið gert í draw.io sjá mynd: 
Við gerum hendi sem á að hreyfa sig nákvæmlega og örugglega og verður mannleg í útliti. Henni verður stjórnað og hreyfð  með  loftþrýstingi. Forritunarmálið sem við notum verður C# með Arduino.
Fyrir utan að vera ógeðslega cool þá mun þetta verkefni hafa stór og góð áhrif á mig og samstarfsmann minn, sem auglýsingargildi og sönnun á hæfni. Þetta var upprunalega planið en vegna ófyrirséðrar seinkunar á sendinguni frá Danmörku þá þurftum við að skipta um verkefni og ákváðum að gera flokkunarvél sem virkar út frá litum. Við erum ennþá með C# sem forritunartungumálið, þetta hefur mikið notagildi, góð æfing í hönnun og samvinnu. Þetta gæti líka verið notað hjá fyrirtækjum til að flokka næstum hvað sem er út frá litum.
\begin{figure}[h]
\includegraphics[scale=.3]{img/system}
\end{figure}
